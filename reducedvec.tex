% TODO: make everything a parameter

\subsection{Általánosan}

Az origóba redukált vektorkettőshöz ki kell számolnunk a pontban ható eredő erőt és az oda eltolt koncentrált erőpárok összegét.

$$[\vec{F}; \vec{M_O}] = \ldots$$

Az eredő erőnél egyszerűen csak összeadjuk az összes rendszerben megjelenő erőt.

$$\vec{F} = \sum_{i} \vec{F_i}$$

A koncentrált erőpároknál már nem csak az összes forgatónyomaték összegét kell figyelembe vennünk, hanem azt is hogy 
% TODO: what exactly do we do here?
tehát erőinket támadáspontjuk helyvektorával kell vektoriálisan megszorozni és ezt hozzáadni az forgatónyomatékok összegéhez.

$$\vec{M_0} = \sum_{j} \vec{M_j} + \sum_{i} \vec{r}_i \times \vec{F_i}$$

\subsection{Megadott adatainkkal}

Az ábra leolvasásához szükséges lesz használnunk az adatokból leolvasható távolságokat.

$$
a = 0.3 (m); \space
b = 0.3 (m); \space
c = 0.4 (m)
$$

$\vec{F_1}$ erőnk vektorának nagysága és értelme meghatározható a megadott adatokból, míg $\vec{F_2}$ és $\vec{F_3}$ erő vektorának értelmét az ábráról leolvasható.

$$
\vec{F_1} = \begin{bmatrix} -2 \\ -1 \\ -1 \end{bmatrix};
\vec{F_2} = \begin{bmatrix} 0 \\ -1 \\ 0 \end{bmatrix};
\vec{F_3} = \begin{bmatrix} 1.3 \\ 0 \\ 0 \end{bmatrix}
$$

...és ezekből az eredő erőt már meg is kaphatjuk. 
$$
\vec{F} = 
\begin{bmatrix} -2 \\ -1 \\ -1 \end{bmatrix} + 
\begin{bmatrix} 0 \\ -1 \\ 0 \end{bmatrix} + 
\begin{bmatrix} 1.3 \\ 0 \\ 0 \end{bmatrix}
= \begin{bmatrix} -0.7 \\ -2 \\ -1 \end{bmatrix} 
$$

Az erők támadáspontjának helyvektorát egyszerűen leolvashatjuk az ábráról.
$$
\vec{r_1} = \begin{bmatrix} a \\ 0 \\ 0 \end{bmatrix} = \begin{bmatrix} 0.3 \\ 0 \\ 0 \end{bmatrix};
\vec{r_2} = \begin{bmatrix} 0 \\ b \\ c \end{bmatrix} = \begin{bmatrix} 0 \\ 0.3 \\ 0.4 \end{bmatrix};
\vec{r_3} = \begin{bmatrix} a \\ b \\ c \end{bmatrix} = \begin{bmatrix} 0.3 \\ 0.3 \\ 0.4 \end{bmatrix}
$$

$\vec{M_1}$ leolvasható a megadott adatokból, míg ${M_2}$-nek csak a nagyságát kapjuk meg de értelmét a mellékelt ábra határozta meg.
$$
\vec{M_1} = \begin{bmatrix} 0.8 \\ 0.5 \\ 0.3 \end{bmatrix};
\vec{M_2} = \begin{bmatrix} 0 \\ 0 \\ -2.1 \end{bmatrix}
$$

A fentiekből már meg is kaphatjuk az összes % TODO: I still don't know what that does

$$
\vec{r_1} \times \vec{F_1} = \begin{bmatrix} a \\ 0 \\ 0 \end{bmatrix}
$$
