\subsection{Általánosan}

Az origóba redukált vektorkettőshöz ki kell számolnunk az eredő erőt és az oda eltolt koncentrált erőpárok összegét.

$$([\vec{F}; \vec{M_O}]_O) = \ldots$$

Az eredő erőt a statika első alapelve alapján megkapjuk ha összeadjuk a rendszerben megjelenő erőket.

$$\vec{F} = \sum_{i} \vec{F_i}$$

A koncentrált erőpároknál már nem csak az összes forgatónyomaték összegét kell figyelembe vennünk, hanem azt is hogy az adott pontba valo redukálás során a különböző erők milyen módon fejtenek ki nyomatékot.

Tehát erőinket támadáspontjuk helyvektorával kell vektoriálisan megszorozni és ezt hozzáadni a forgatónyomatékok összegéhez.

$$\vec{M_0} = \sum_{j} \vec{M_j} + \sum_{i} \vec{r}_i \times \vec{F_i}$$

\subsection{Megadott adatainkkal}

Az ábra leolvasásához szükséges lesz használnunk az adatokból leolvasható távolságokat.
\begin{align}
	a = 0.3 \, [\si{m}] \\ 
	b = 0.3 \, [\si{m}] \\
	c = 0.4 \, [\si{m}]
\end{align}

$\vec{F_1}$ erőnk vektorának nagysága és értelme meghatározható a megadott adatokból, míg $\vec{F_2}$ és $\vec{F_3}$ erő vektorának értelmét az ábráról leolvashatjuk.

\begin{align}
	\vec{F_1} = \begin{bmatrix} -2 \\ -1 \\ -1 \end{bmatrix} [\si{kN}] \\
	\vec{F_2} = \begin{bmatrix} 0 \\ -1 \\ 0 \end{bmatrix} [\si{kN}] \\
	\vec{F_3} = \begin{bmatrix} 1.3 \\ 0 \\ 0 \end{bmatrix} [\si{kN}]
\end{align}

\break

...és ezekből az eredő erőt már meg is kaphatjuk. 
$$
\vec{F} = 
\begin{bmatrix} -2 \\ -1 \\ -1 \end{bmatrix} + 
\begin{bmatrix} 0 \\ -1 \\ 0 \end{bmatrix} + 
\begin{bmatrix} 1.3 \\ 0 \\ 0 \end{bmatrix}
= \begin{bmatrix} -0.7 \\ -2 \\ -1 \end{bmatrix} [\si{kN}]
$$

Az erők támadáspontjának helyvektorát egyszerűen leolvashatjuk az ábráról.
\begin{gather}
	\vec{r_1} = \begin{bmatrix} a \\ 0 \\ 0 \end{bmatrix} = \begin{bmatrix} 0.3 \\ 0 \\ 0 \end{bmatrix} [\si{m}] \\
\vec{r_2} = \begin{bmatrix} 0 \\ b \\ c \end{bmatrix} = \begin{bmatrix} 0 \\ 0.3 \\ 0.4 \end{bmatrix} [\si{m}] \\
\vec{r_3} = \begin{bmatrix} a \\ b \\ c \end{bmatrix} = \begin{bmatrix} 0.3 \\ 0.3 \\ 0.4 \end{bmatrix} [\si{m}]
\end{gather}

$\vec{M_1}$ leolvasható a megadott adatokból, míg ${M_2}$-nek csak a nagyságát kapjuk meg de értelmét a mellékelt ábra határozta meg és már össze is adhatjuk őket.
\begin{align}
\vec{M_1} = \begin{bmatrix} 0.8 \\ 0.5 \\ 0.3 \end{bmatrix} [\si{kNm}] \\
\vec{M_2} = \begin{bmatrix} 0 \\ 0 \\ -2.1 \end{bmatrix} [\si{kNm}] \\
\sum_{j} \vec{M_j} = \begin{bmatrix} 0.8 \\ 0.5 \\ -1.8 \end{bmatrix} [\si{kNm}]
\end{align}

A fentiekből már meg is kaphatjuk az összes erő adott pontunkra való nyomatékát.

\begin{align}
\vec{r_1} \times \vec{F_1} = \begin{bmatrix} 0 \\ 0.3 \\ -0.3 \end{bmatrix} [\si{kNm}] \\
\vec{r_2} \times \vec{F_2} = \begin{bmatrix} 0.4 \\ 0 \\ 0 \end{bmatrix} [\si{kNm}] \\
\vec{r_3} \times \vec{F_3} = \begin{bmatrix} 0 \\ 0.52 \\ -0.39 \end{bmatrix} [\si{kNm}] \\
\sum_{i} \vec{r}_i \times \vec{F_i} = \begin{bmatrix} 0.4 \\ 0.82 \\ -0.69 \end{bmatrix} [\si{kNm}]
\end{align}

...és ezekkel a redukált nyomatékvektort ki is számíthatjuk.
$$\vec{M_0} = \sum_{j} \vec{M_j} + \sum_{i} \vec{r}_i \times \vec{F_i} = \begin{bmatrix} 0.8 \\ 0.5 \\ -1.8 \end{bmatrix} + \begin{bmatrix} 0.4 \\ 0.82 \\ -0.69 \end{bmatrix} = \begin{bmatrix} 1.2 \\ 1.32 \\ -2.49 \end{bmatrix} [\si{kNm}]$$

Így a redukált vektorkettősünk már össze is állítható.
\begin{center}
\fbox{
	\begin{varwidth}{\textwidth}
		$$([\vec{F}; \vec{M_O}]_O) = \left(\begin{bmatrix} \begin{bmatrix} -0.7 \\ -2 \\ -1 \end{bmatrix};\begin{bmatrix} 1.2 \\ 1.32 \\ -2.49 \end{bmatrix}\end{bmatrix}_O\right) [\si{kN};\si{kNm}]$$
	\end{varwidth}
}
\end{center}
