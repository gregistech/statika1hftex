
\subsection{Általánosan}

Egy nyomaték hatását egy adott tengelyre megkaphatjuk önmaga és a tengely skaláris szorzatával. Ezen nyomaték párhuzamos lesz az $\vec{F}$ erővel.

$$M_f = \vec{M_o} \cdot \vec{f}$$

A legegyszerűbb egy egységvektort készítenünk az adott erőnkből.

$$\vec{F} \rightarrow \vec{f}: \vec{f} = \frac{\vec{F}}{|\vec{F}|}$$

$$F = |\vec{F}| = \sqrt{F_x^2 + F_y^2 + F_z^2}$$

\subsection{Megadott adatainkkal}

Eddigi számolásaink alapján mindent be is tudunk helyettesíteni.

$$|\vec{F}| = \sqrt{F_x^2 + F_y^2 + F_z^2} = \sqrt{(-0.7)^2 + (-2)^2 + (-1)^2}$$ 
$$= \sqrt{0.49 + 4 + 1} = \sqrt{5.49} = \frac{3\sqrt{61}}{10} \approx 2.343 \, [\si{N}]$$

$$\vec{f} = \frac{\vec{F}}{|\vec{F}|} = \frac{\begin{bmatrix} -0.7 \\ -2 \\ -1 \end{bmatrix}}{\frac{3\sqrt{61}}{10}} = \begin{bmatrix} -0.298 \\ -0.853 \\ -0.426 \end{bmatrix} [\si{1}]$$

\begin{center}
\fbox{
	\begin{varwidth}{\textwidth}
		$$M_f = \vec{M_o} \cdot \vec{f} = \begin{bmatrix} 1.2 \\ 1.32 \\ -2.49 \end{bmatrix} \cdot \begin{bmatrix} -0.298 \\ -0.853 \\ -0.426 \end{bmatrix} = -0.423 \, [\si{kNm}]$$
	\end{varwidth}
}
\end{center}

